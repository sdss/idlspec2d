%%%%%%%%%%%%%%%%%%%%%%%%%%%%%%%%%%%%%%%%%%%%%%%%%%%%%%%%%%%%%%%%%%%%%%%%%

% Definitions
%\pagestyle{empty}
\documentclass[12pt,preprint]{aastex}
%\documentstyle[emulateapj,danonecolfloat]{article}
%\usepackage{emulateapj,danonecolfloat}
\usepackage{rotating}

%------------------------------------------------------------------------------
% DEFINITIONS
%------------------------------------------------------------------------------
%
% Equations:
%
\def\BE{\begin{equation}}
\def\BEL#1{\begin{equation}\label{#1}}
\def\EE{\end{equation}}
%
% Random stuff:
%
\newcommand{\etal}{{\it et al.}~}
\newcommand{\LOGTEN}{{\log_{10}}}
%
% Units (math mode)
%
\newcommand{\Ang}{{\rm ~\AA}}
\newcommand{\cm}{{\rm ~cm}}
\newcommand{\dfdegree}{^\circ}
\newcommand{\degree}{^\circ}
\newcommand{\ergs}{{\rm ~erg~s}^{-1}}
\newcommand{\ergscmang}{{\rm ~erg~s}^{-1}{\rm cm}^{-2}{\rm\AA}^{-1}}
\newcommand{\ergcmHzsr}{{\rm ~erg~cm}^{-2}{\rm s}^{-1}
            {\rm Hz}^{-1}{\rm sr}^{-1}}
\newcommand{\Jy}{{\rm ~Jy}}
\newcommand{\JypSr}{{\rm ~Jy~sr^{-1}}}
\newcommand{\GHz}{{\rm ~GHz}}
\newcommand{\Hz}{{\rm ~Hz}}
\newcommand{\K}{{\rm ~K}}
\newcommand{\kms}{{\rm ~km/s}}
\newcommand{\MAG}{{\rm ~mag}}
\newcommand{\microK}{\mu{\rm K}}
\newcommand{\MJy}{{\rm ~MJy}}
\newcommand{\mJy}{{\rm ~mJy}}
\newcommand{\MJypSr}{{\rm ~MJy~sr^{-1}}}
\newcommand{\nWpMMSr}{{\rm ~nW~m}^{-2}{\rm sr}^{-1}}
\newcommand{\mK}{\rm ~mK}
\newcommand{\mm}{{\rm ~mm}}
\newcommand{\nMgy}{{\rm nMgy}}
\newcommand{\pc}{{\rm ~pc}}
\newcommand{\pix}{{\rm ~pix}}
\newcommand{\s}{{\rm ~s}}

%------------------------------------------------------------------------------
% TITLE PAGE
%------------------------------------------------------------------------------

\begin{document}

\title{The Sloan Digital Sky Survey Spectroscopic Reduction Pipeline}

\author{
S. Burles\altaffilmark{\ref{MIT}}
and D. J. Schlegel\altaffilmark{\ref{Princeton}}
}
\altaffiltext{1}{Massachussetts Institute of Technology,
Boston, MA ????? \label{MIT}}
\altaffiltext{2}{Princeton University Observatory, Peyton Hall, Princeton,
NJ 08544 \label{Princeton}}

%------------------------------------------------------------------------------
% ABSTRACT
%------------------------------------------------------------------------------

\begin{abstract}
The Sloan Digital Sky Survey (SDSS) fiber-fed spectrographs have been used
to obtain spectra for $600,000$ galaxies, QSOs and stars.
We present the algorithms for reducing the raw data to
wavelength-calibrated, flux-calibrated one-dimensional spectra.
The wavelength calibration is shown to have an accuracy of $3 \kms$,
and the broad-band spectrophotometry has an accuracy of better than $10\%$.

%------------------------------------------------------------------------------
% SUBJECT HEADINGS
%------------------------------------------------------------------------------
\emph{Subject headings: }
???
\end{abstract}

%------------------------------------------------------------------------------
% INTRODUCTION
%------------------------------------------------------------------------------
\section{INTRODUCTION}
\label{sec_intro}
The Sloan Digital Sky Survey (SDSS) is a 5-band photometric survey of
approximately $10,000$ square degrees of the Northern sky and a concurrent
redshift survey of up to a million galaxies and 100,000 quasars
selected from the imaging survey \cite{york00}.  The primary 
science goals of the project are to provide the data to investigate the
large scale structure of the Universe and other extragalactic science.

The data reduction and products described herein are very similar to
those produced by the SDSS, with the most significant differences
in our treatment of spectro-photometric calibrations.
The data are released
with this paper and may be accessed via the WWW.\footnote{Full details
are at \texttt{http://spectro.princeton.edu}}
Future data releases will also be accessible at this site. 

%------------------------------------------------------------------------------
%------------------------------------------------------------------------------
\section{OPTIMAL EXTRACTION}
\label{sec_extract}


%------------------------------------------------------------------------------
%------------------------------------------------------------------------------
\section{FLUX CALIBRATION}
\label{sec_fluxing}

The SDSS spectra are presented as spectro-photometric quantitites,
meaning that they have been converted to physical flux densities
as seen above the Earth's atmosphere.
There are many difficulties inherent in a fiber-fed spectrograph,
especially one with such a large field of view.
The problem is complicated by changes in the flat-field lamp spectra,
the throughput through some of the optical elements, changing
atmospheric transparency, and guiding errors.
Despite these problems, the spectrophotometry is accurate to
several percent on all wavelength scales.

Flux-calibration for the two spectrographs are treated almost completely
independently.  This is appropriate because the collimator optics
and CCDs are different.  Only the telescope optics are in common, and the
atmospheric extinction and guiding errors are continuous across a plate.
In general, the 320 fibers for spectrograph \#1 are plugged for those
objects south of the centerline of the field of view, and the 320 fibers
for spectrograph \#2 are plugged for the north.

% The spectra are converted from raw counts (ADUs) to physical
% units ($\ergs$).
% $ 100 \ergscmang$

We first convert from ADUs to physical flux units under the
assumption that there are no guiding or object centroiding errors.
A mean relative response between fibers has already been corrected
using the quartz flat-field lamps (see Section ???).
Because of this flat-fielding, the remaining conversion from ADUs
to physical units should be identical for each spectrum on the 
same CCD, except for small, known differences from different pixel scales
and airmass.

The SDSS spectrophotometry is based upon our understanding
of F star subdwarfs.  These stars are the easiest main sequence
stars to theoretically model because ...???
For these reasons, we target 16 of these stars on each plate,
typically 8 on each spectrograph.


% See http://www.astro.princeton.edu:81/sdss-target/msg.725.html

We select relatively bright F dwarfs as spectrophotometric 
standards.  Ideally, we want the low-metallicity, halo stars in order 
to have spectra that have atmospheric opacities that are easy to model.
The F stars are chosen to satisfy the following color criteria:
$$ 0.6 < u-g < 1.2 $$
$$ 0.0 < g-r < 0.6 $$
$$ g-r > 0.75 * (u-g) - 0.45 $$
as shown in Figure ???.
This selects stars that are on the blue edge of the main
sequence in (u-g) color.  At high Galactic latitudes,
??? stars per square degree fall in this color box.
Targetting priority is given to
those stars with colors closest to the following:
$$ u-g = 0.80 $$
$$ g-r = 0.30 $$
$$ r-i = 0.10 $$
This is meant to select those stars closest in color to BD +17 4708,
which is computed to have a similar color (0.93, 0.28, 0.01, courtesy
of David Hogg, priv. comm.).
From these stars, we select 8 on each plate in the magnitude
range $15.5 < g < 17.0$ (which we call SPECTROPHOTO\_STD stars)
and 8 in the magnitude range $17.0 < g < 18.5$ (which we call
REDDEN\_STD stars).
These magnitude limits are chosen to avoid nearing saturation
of the CCDs on the bright end, but to still ensure good S/N
on the faint end.
Assuming an absolute magnitude of 4.3 for these stars, this
implies distances of $1.7 < d < 3.5$ kpc for the SPECTROPHOTO\_STD
stars and $3.5 < d < 6.9$ kpc for the REDDEN\_STD stars.
Note that the color-selection and magnitude cuts are done on
``instrinsic'' magnitudes, meaning that we first apply corrections
for SFD extinction under the assumption that these stars are
completely behind the Galactic dust.

The spectra for these stars are compared to the model spectra
from the SPECTRUM code of Gray \& Corbally (1994) as updated
in Gray \etal\ (2001).
We constructed a grid of 90 models with effective temperatures
from $5000$ to $7000 \K$ (spaced every $250 \K$), [Fe/H] values
from -2 to 0 (spaced every 0.5), and gravities of $\log g=4$
and $\log g=4.5$.  These models are then ``observed'' as they
would be seen in our spectra.  We convolve the models with
the instrumental dispersion as a function of wavelength
(see Section ???), and sample the results at the observed wavelengths.
Since there are typically three exposures in the two spectrographs
(blue and red) for each star, this means we re-generate the full
grid of models for six observations of each star.
We perform $\chi^2$ minimizations between these models and the
6 spectra simultaneously.  We use the observed errors per pixel,
which are independent because we have done no resampling of the data.
All spectra and models are first median-filtered with a scale of $7000\kms$,
and the regions of atmospheric absorption in the red are masked
for the purpose of spectral typing our stars.
A velocity in the range $-700 < v < +700 \kms$ is fit
by comparing to the model with $T_{eff} = 6000\K$, Fe/H=$-1.5$,
$\log g=4$.  All the models are redshifted with this best-fit velocity,
and the resulting model with the smallest $\chi^2$ is chosen
as the best-fit model.  We reject any stars with a reduced $\chi^2$
larger than 2.0 in the full spectrum, with a reduced $\chi^2$
larger than 2.0 in the wavelength regions of the strong stellar
absorption lines, or with a median S/N less than 2.0 in
those absorption line regions.
We also reject stars if more than $20\%$ of the pixels are bad
in any of the 3 or more observations, which typically only happens
if a fiber is broken or the spectrum falls on bad columns on the CCD.
Figure ??? shows the distribution of parameters that we have fit
for the ??? flux-calibration stars through Data Release 5.

Each flux-calibration star provides a means of converting our
observed spectra to physical units.  
In practice, a single star is not sufficient because of the
slight undersampling in wavelength, bad pixels in any
particular observation, uncertainties in our stellar typing,
and the desire to achieve better S/N in the flux-calibration
vector than one star provides.  Each of these F star spectra
are ratioed with the best-fit model at the best-fit redshift,
reddened with the SFD reddening maps, and scaled to the SDSS r-band
magnitudes.  The reddening is applied using an $R_V=3.1$ extinction
law from O'Donnell (1994), and assuming that these stars reside
behind all the Galactic dust.
These ratios are called the ``flux calibration vectors.''

The set of flux-calibration vectors for an entire plate are combined
using B-splines to compute an average flux-calibration vector per CCD.
For the red CCDs, we fit 2-dimensional B-splines where the fits are
allowed to vary linearly with airmass at each spline break point
in wavelength.  These airmass terms are disabled if the plate spans
less than 0.1 in airmass, or if there are fewer than 3 good stars.
Because the B-spline fits can be dominated by
spectrophotometry errors of low order in wavelength, and by
atmospheric throughput variations, each fluxing
vector is first corrected to the mean vector using cubic polynomials.
The B-spline fits are then evaluated on an object-by-object basis,
using the airmass of each fiber, and multiplying back in the mean
cubic correction for the F stars each exposure.
The mean of those cubic corrections are computed such that they
are consistent for the same objects between the blue and red CCDs
on each exposure.

At this point, the spectra can be flux-calibrated to the mean
F star response in each exposure, including correction for
atmospheric absorption on an object-by-object basis.
The problem would be solved at this point if there were
no centroiding errors, guiding errors, differential refraction,
or transparency variations across the 3-degree field.
...???




%------------------------------------------------------------------------------
\subsection{GALACTIC REDDENING}

The SDSS spectra are presented as observed above the Earth's
atmosphere, but not above the Galaxy's atmosphere.
If one is studying distant stars or extragalactic objects,
one may wish to remove the effects of Galactic extinction
on these spectra.  It is suggested that one use the SFD reddening
maps and the extinction curves from either Cardelli \etal\ (1989)
or O'Donnell (1994) with $R_V=3.1$.  The differences between the
two extinction curves is less than $1\%$, but we choose the O'Donnell
curves as the more modern values.  For the optical wavelengths
of interest, the relevant formulae are
assembled from those two papers below:
$$ {A_\lambda \over A_V} = a(y) + {b(y)\over R_V} $$
$$ y \equiv {10^4\AA \over \lambda} - 1.82 $$
$$ a(y) = 1.000 + 0.104 y - 0.609 y^2 + 0.701 y^3 + 1.137 y^4
  - 1.718 y^5 - 0.827 y^6 + 1.647 y^7 - 0.505 y^8 $$
$$ b(y) = 1.952 y + 2.908 y^2 - 3.989 y^3 - 7.985 y^4
  + 11.102 y^5 + 5.491 y^6 - 10.805 y^7 + 3.347 y^8 $$

{\bf Need CCM above 9091 Ang!!!???}

Note that the extinction corrections should be applied in the
observed wavelength frame, before any de-redshifting of extragalactic objects.


%------------------------------------------------------------------------------
% CONCLUSIONS
%------------------------------------------------------------------------------
%\section{SUMMARY}

\newpage
%------------------------------------------------------------------------------
% REFERENCES
%------------------------------------------------------------------------------

\bibliographystyle{unsrt}
\bibliography{gsrp}

\begin{thebibliography}{DUM}

%Gray's code SPECTRUM is
%available http://www.acs.appstate.edu/dept/physics/spectrum/spectrum.html
\bibitem[Gray et al. 2001]{gray01}
Gray, R.\ O., Graham, P.\ W., \& Hoyt, S.\ R. 2001, \aj, 121, 2159

\bibitem[Gray \& Corbally 1994]{gray94}
Gray, R.\ O., \& Corbally, C.\ J. 1994, \aj, 107, 742

\bibitem[O'Donnell 1994]{odonnell94} 
O'Donnell, J.\ E. 1994, \apj, 422, 158

\bibitem[Schlegel, Finkbeiner \& Davis 1998 (SFD)]{sfd98} 
Schlegel, D.\ J., Finkbeiner, D.\ P., \& Davis M.\ 1998, \apj, 500, 525 (SFD)

\bibitem[York et al. 2000]{york00}
York, D.\ G. \etal\ 2000, \aj, 120, 1579

\end{thebibliography}
\clearpage

%------------------------------------------------------------------------------
%------------------------------------------------------------------------------
\appendix
\section{B-SPLINES}
\label{bsplines}

What the heck is a B-spline?

%------------------------------------------------------------------------------
%------------------------------------------------------------------------------
\appendix
\section{DATA MODEL}
\label{bsplines}

What the heck is a B-spline?

%------------------------------------------------------------------------------

\end{document}
