

The object extraction lists 5 steps in header:

\begin{enumerate}
\item{Locate bright fibers (aka whopping fibers)}
\item{Assess sky fiber levels}
\item{Commence iterative Optimal extraction }
\item{Shift wavelength solution to match sky lines (describe in wavelength solution)}
\item{Sky subtract}

\end{enumerate}

Note: Flux correction and Telluric correction have been moved (a long time ago).


\subsection{Sky Subtraction}

A global sky model is fit using all available fiber spectra assigned as 
"SKY" by target selection.  These fibers are set on blank fields, and are 
used to estimate the sky level as a function of wavelength and CCD position 
in individual exposures.   Each plate is nominally assigned 32 of its 640 
fibers as "SKY", with the focal plane positions chosen to allow half of the 
sky fibers to reach each spectrograph: 1 fiber per bundle of 20 is assigned to 
"SKY".

The sky subtraction is done after the data have been extracted from 
two-dimensional counts per CCD pixel to one-dimensional "corrected" electron 
counts as a function of vacuum heliocentric wavelengths and fiber number.
The flat-field "correction" removes differences due to surface brightness 
changes due to varying spectral dispersion in different fibers.

A single subroutine is called three consecutive times to arrive at the final 
model description for the sky background.  The first call uses all available sky
fibers that were not masked as an entire fiber (e.g. by the fibermask bits).
If only one good skyfiber is found, the routine logs an "ABORT" and returns.
A median count level is summarized for the selected sky spectra, and the 5% 
and 95% levels are reported for each science exposure (on each camera).
A single one-dimensional sky model is created by a b-spline fit to the airmass 
corrected sky spectra (See appendix b-spline for a continued description).
The airmass associated with the exposure mid-point (in time) is divided out 
of each sky spectrum (and errors scaled accordingly).  The differential airmass
across a single SDSS plate is the largest source of residuals after the dominant
wavelength residuals.  Individual pixels are masked if they were divided by a 
low flat field vector or a bad pixel was detected nearby (clarify?).
The number of discarded pixels before sky modeling is logged.

The sky




     

